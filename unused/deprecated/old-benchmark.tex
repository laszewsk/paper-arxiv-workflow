\begin{comment}
Figures~\ref{fig:six-graphs} show accuracy values of NNSE with
various epoch values for training and validation. We chose here 2, 30,
and 70 epochs.  We have then created tables for the NNSE values for
each of the epochs and sorted them by value. The higher the NNSE
accuracy value is the better. We have created tables for the 2, 30,
and 70 epoch hyper-parameters. In addition, we highlighted in the table
the best training value with red, the second best with blue, and the
third best with three. We also highlighted the corresponding value in
the inference column that corresponds to the hyperparameter for the
time period used in the input and output parameters. While we see for
the 2 and 30 epoch cases the same order for these three values, we see
a different order for the 70 epoch case. This can be explained through
overfitting. Overall we find that the best accuracy can be found for
\end{comment}

\begin{comment}
  \centering
    \includegraphics[width=0.70\columnwidth]{images/performance-projection.png}
    \caption{Performance projection. }
    \label{fig:performance-projection}
  \end{comment}
    


\begin{comment}

\begin{figure}[htb]

  \begin{center}

     \begin{minipage}[b]{0.45\textwidth}
       \includegraphics[width=1.0\linewidth]{images/2_training-MSE-and-NNSE.pdf}
        {\bf (A)} MSE and NNSE - 2 epochs training.
    \end{minipage}
     \ \
     \begin{minipage}[b]{0.45\textwidth}
        \includegraphics[width=1.0\linewidth]{images/2_validation-MSE-and-NNSE.pdf}
        {\bf (B)}  MSE and NNSE - 2 epochs validation.
     \end{minipage}

     \begin{minipage}[b]{0.45\textwidth}
        \includegraphics[width=1.0\linewidth]{images/30_training-MSE-and-NNSE.pdf}
        {\bf (C)} MSE and NNSE - 30 epochs training.
     \end{minipage}
     \ \
     \begin{minipage}[b]{0.45\textwidth}
        \includegraphics[width=1.0\linewidth]{images/30_validation-MSE-and-NNSE.pdf}
        {\bf (D)} MSE and NNSE - 30 epochs validation.
     \end{minipage}

     \begin{minipage}[b]{0.45\textwidth}
        \includegraphics[width=1.0\linewidth]{images/70_training-MSE-and-NNSE.pdf}
        {\bf (E)} MSE and NNSE - 70 epochs training.
     \end{minipage}
     \ \
     \begin{minipage}[b]{0.45\textwidth}
        \includegraphics[width=1.0\linewidth]{images/70_validation-MSE-and-NNSE.pdf}
        {\bf (F)}  MSE and NNSE - 70 epochs validation.
     \end{minipage}
\end{center}

     \caption{NNSE and MSE values for training and validation for epochs 2 (A, B), 30 (C, D), 70 (E, F).}
     \label{fig:six-graphs}
\end{figure}

\end{comment}

%we have finalized the EQ code but want to make absolutely sure that we
%look at the correct values for the scientific comparison.
%
%This also requires a small sentence to each variable. Could we have a
%small meeting and I take then some notes on what these values are
%tomorrow. I will then add the explanations to the MLCommons EQ benchmark
%policy document.
%
%I just want to make sure I understand over which domain we average and
%sum up.
%
%Also if we were to just do one value (just in case they ask, I think we
%would use the summed up total right. However I think it is better to
%keep all of them.)
%
%Also I forgot what the +26 refers to
%
%I think something like this is almost correct, but we need to +26
%explanation and get verification from you.
%
%The Magnification based on a years worth of back data, while looking two
%weeks ahead + 26 what?

\begin{comment}
\begin{table}[p]

  \caption{Training and validation with time-based hyperparameters
    sorted by NNSE accuracy. The table includes the best two
    values highlighted in the training and validation results to
    showcase the accuracy of the validation. In the validation,
    we see that the best value for training is in rank four for the
    validation. The number of Epochs for this experiment is 2.
    26 is half of 52 and so 26 2-week intervals is a year.}
  \label{tab:training-2}

  \renewcommand{\arraystretch}{1.2}
  \begin{center}
    {\footnotesize
\begin{tabular}{|r|rl||rl|}
  \hline
{\bf Rank} & \multicolumn{2}{c||}{\bfseries Training} & \multicolumn{2}{c|}{\bfseries Validation}  \\
     &   {\bf NNSE} & {\bf Hyperparameters} & {\bf NNSE} & {\bf Hyperparameters} \\
\hline
1 & \color{red}  0.521800 & \color{red} Now 2wk+26AVG & \color{red} 0.502300 & \color{red} Now 2wk+26AVG \\
2 & \color{blue} 0.506800 & \color{blue} 2 weeks Now & \color{blue} 0.470100 & \color{blue} 2 weeks Now \\
3 & \color{teal} 0.429900 & \color{teal} Now 2wk+13AVG & \color{teal} 0.412700 & \color{teal} Now 2wk+13AVG \\
4 & 0.405600 & Now 2wk+7AVG & 0.379700 & Now 2wk+7AVG \\
5 & 0.302500 & 3M 2wk+26AVG & 0.295800 & 3M 2wk+26AVG \\
6 & 0.278800 & 3M 2wk+13AVG & 0.270300 & 1Y 2wk+26AVG \\
7 & 0.251700 & 6M 2wk+26AVG & 0.266200 & 3M 2wk+13AVG \\
8 & 0.251600 & 1Y 2wk+26AVG & 0.264400 & 6M 2wk+26AVG \\
9 & 0.243000 & 3M 2wk+7AVG & 0.249500 & 1Y 2wk+7AVG \\
10 & 0.235800 & 1Y 2wk+7AVG & 0.238200 & 1Y 2wk+13AVG \\
11 & 0.233000 & 3 Months Back & 0.228900 & 3M 2wk+7AVG \\
12 & 0.232600 & 1Y 2wk+13AVG & 0.219700 & 3 Months Back \\
13 & 0.201600 &  6 Months Back & 0.204500 & Year Back \\
14 & 0.197000 & 6M 2wk+13AVG & 0.201600 & 6M 2wk+13AVG \\
15 & 0.192700 &  6M 2wk+7AVG &  0.201000 & 6 Months Back \\
16 &  0.191300 & Year Back & 0.195200 & 6M 2wk+7AVG \\
\hline
\end{tabular}
}
\end{center}

%\end{table}

%\begin{table}[htb]

  \caption{Training and validation with time-based hyperparameters
    sorted by NNSE accuracy. The table includes the best two
    values highlighted in the training and validation results to
    showcase the accuracy of the validation. In the validation,
    we see that the best value for training is in rank four for the
    validation. The number of Epochs for this experiment is 30.
  }
  \label{tab:training-30}

  \renewcommand{\arraystretch}{1.2}
  \begin{center}
        {\footnotesize
\begin{tabular}{|r|rl||rl|}
\hline
{\bf Rank} &
\multicolumn{2}{c||}{\bfseries Training} &
\multicolumn{2}{c|}{\bfseries Validation} \\
     {\bf NNSE} &
     {\bf Hyperparameters} &
     {\bf NNSE} &
     {\bf Hyperparameters} \\
\hline
1 & \color{red} 0.586000 & \color{red} Now 2wk+26AVG & \color{red} 0.484100 & \color{red} Now 2wk+26AVG \\
2 & \color{blue} 0.559200 & \color{blue} Now 2wk+13AVG & \color{blue} 0.453000 & \color{blue} Now 2wk+13AVG \\
3 & \color{teal} 0.517900 & \color{teal} Now 2wk+7AVG & \color{teal}0.409300 & \color{teal} Now 2wk+7AVG \\
4 & 0.488800 & 2 weeks Now & 0.398900 & 3M 2wk+26AVG \\
5 & 0.450800 & 3M 2wk+26AVG & 0.384100 & 2 weeks Now \\
6 & 0.418900 & 6M 2wk+26AVG & 0.374500 & 6M 2wk+26AVG \\
7 & 0.394600 & 3M 2wk+13AVG & 0.343400 & 3M 2wk+13AVG \\
8 & 0.341800 & 6M 2wk+13AVG & 0.302800 & 6M 2wk+13AVG \\
9 & 0.330300 & 3M 2wk+7AVG & 0.291900 & 3M 2wk+7AVG \\
10 & 0.319600 & 1Y 2wk+26AVG & 0.290100 & 1Y 2wk+26AVG \\
11 & 0.208100 & 6M 2wk+7AVG & 0.186200 & 6M 2wk+7AVG \\
12 & 0.171600 & 1Y 2wk+13AVG & 0.153600 & 1Y 2wk+13AVG \\
13 & 0.089700 & 3 Months Back & 0.090400 & 3 Months Back \\
14 & 0.082900 & 1Y 2wk+7AVG & 0.076500 & 1Y 2wk+7AVG \\
15 & 0.069500 &  6 Months Back & 0.070300 & 6 Months Back \\
16 &  0.047600 & Year Back & 0.050500 & Year Back \\
\hline
\end{tabular}
}

\end{center}
\end{table}


\begin{table}[htb]

  \caption{Training and validation with time-based hyperparameters
    sorted by NNSE accuracy. The table includes the best two
    values highlighted in the training and validation results to
    showcase the accuracy of the validation. In the validation,
    we see that the best value for training is in rank four for the
    validation. The number of Epochs for this experiment is 70.}
  \label{tab:training-70}

  \renewcommand{\arraystretch}{1.2}
  \begin{center}
        {\footnotesize
\begin{tabular}{|r|rl||rl|}
  \hline
{\bf Rank} & \multicolumn{2}{c||}{\bfseries Training} & \multicolumn{2}{c|}{\bfseries Validation} \\
     &   {\bf NNSE} & {\bf Hyperparameters} & {\bf NNSE} & {\bf Hyperparameters} \\
              \hline
 01 & \color{red} 0.490400 & \color{red} Now 2wk+26AVG & \color{red} 0.412500 & \color{red} Now 2wk+26AVG \\
 02 & \color{blue} 0.465800 & \color{blue}  Now 2wk+13AVG & \color{blue} 0.385800 & \color{blue} Now 2wk+13AVG \\
 03 & \color{teal} 0.441300 &  \color{teal} Now 2wk+7AVG & 0.370700 & 3M 2wk+26AVG \\
 04 & 0.425700 & 2 weeks Now & \color{teal} 0.359400 & \color{teal} Now 2wk+7AVG \\
 05 & 0.421000 & 3M 2wk+26AVG & 0.344500 & 2 weeks Now \\
 06 & 0.385200 & 3M 2wk+13AVG & 0.333000 & 3M 2wk+13AVG \\
 07 & 0.335400 & 3M 2wk+7AVG & 0.293500 & 1Y 2wk+26AVG \\
 08 & 0.318800 & 1Y 2wk+26AVG & 0.291100 & 3M 2wk+7AVG \\
 09 & 0.295900 & 6M 2wk+26AVG & 0.269700 & 6M 2wk+26AVG \\
 10 & 0.252100 & 6M 2wk+13AVG & 0.235400 & 6M 2wk+13AVG \\
 11 & 0.153500 & 1Y 2wk+13AVG & 0.142800 & 1Y 2wk+13AVG \\
 12 & 0.138400 & 6M 2wk+7AVG & 0.133700 & 6M 2wk+7AVG \\
 13 & 0.105300 & 6 Months Back & 0.102200 & 6 Months Back \\
 14 & 0.083100 & 1Y 2wk+7AVG & 0.084300 & 1Y 2wk+7AVG \\
 15 & 0.073500 & Year Back & 0.071200 & Year Back \\
 16 & 0.067400 & 3 Months Back & 0.069800 & 3 Months Back \\
\hline
\end{tabular}
}
  \end{center}


\end{table}

\end{comment}


%%%%%%%

\begin{comment}

\begin{figure}[p]

  \begin{center}
     \begin{minipage}[t]{0.65\textwidth}
        \includegraphics[width=1.0\linewidth]{images/NNSE-all-epochs-training}
        {\bf (A)} NNSE for training.
     \end{minipage}
  \end{center}
  \ \
  \begin{center}
     \begin{minipage}[t]{0.65\textwidth}
        \includegraphics[width=1.0\linewidth]{images/NNSE-all-epochs-validation}
        {\bf (B)} NNSE for validation.
     \end{minipage}
  \end{center}

  \caption {NNSE comparison}
  \label{fig:NNSE-comparison-v100}

\end{figure}

\end{comment}

\begin{figure}[p]

  \begin{center}
     \begin{minipage}[t]{0.65\textwidth}
        \includegraphics[width=1.0\linewidth]{images/frequency_nnse_histogram_70_training.pdf}
        {\bf (A)} NNSE for training.
     \end{minipage}
  \end{center}
  \ \
  \begin{center}
     \begin{minipage}[t]{0.65\textwidth}
        \includegraphics[width=1.0\linewidth]{images/frequency_nnse_histogram_70_validation.pdf}
        {\bf (B)} NNSE for validation.
     \end{minipage}
  \end{center}

  \caption {NNSE comparison}
  \label{fig:NNSE-history-a100}

\end{figure}